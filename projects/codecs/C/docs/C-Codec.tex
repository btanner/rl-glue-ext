\documentclass[11pt]{article}

\usepackage{graphics}
\usepackage{url}
\usepackage{verbatim}
\usepackage{fullpage}
\usepackage{html}
%I don't like my paragraphs indented because we have lots of 1-line paragraphs with URLs or code after them
\setlength{\parindent}{0in} 
%I like space between my paragraphs
\setlength{\parskip}{10pt} 

\title{RL-Glue C/C++ Codec 2.0 Manual }
\author{Brian Tanner :: brian@tannerpages.com}
\date{}                                        

\begin{document}
\maketitle
\tableofcontents

\section{Introduction}

This document describes how to use the C/C++ RL-Glue Codec, a software library that provides socket-compatibility with the RL-Glue Reinforcement Learning software library.  

For general information and motivation about the RL-Glue\footnote{\url{http://glue.rl-community.org/}} project, please refer to the documentation provided with that project.

This codec will allow you to create agents, environments, and experiment programs in C and/or C++.

This software project is licensed under the Apache-2.0\footnote{\url{http://www.apache.org/licenses/LICENSE-2.0.html}} license. We're not lawyers, but our intention is that this code 
should be used however it is useful.  We'd appreciate to hear what you're using it for, and to get credit if appropriate.

This project has a home here:\\
\url{http://rl-glue-ext.googlecode.com}

This document should be used in combinations with the technical manual for RL-Glue 3.0.  The RL-Glue 3.0 technical manual explains the data types, function prototypes, etc. for the main C/C++ RL-Glue implementation that is used
by this codec.  The RL-Glue 3.0 technical manual can be found in the \texttt{docs} section of the RL-Glue project.

\subsection{Software Requirements}
This project requires that RL-Glue has been installed on your computer.  It has no additional requirements beyond RL-Glue: nothing more exotic than a C compiler, Make, etc.  This codec uses a configure script that was created by GNU Autotools\footnote{\url{http://sources.redhat.com/autobook/}}, so it should compile and run without problems on most *nix platforms (Unix, Linux, Mac OS X, Windows using CYGWIN\footnote{\url{http://www.cygwin.com/}}). 

\subsection{Getting the Codec}
The codec can be downloaded either as a tarball or can be checked out of the subversion repository where it is hosted.

The tarball distribution can be found here:\newline
\url{http://code.google.com/p/rl-glue-ext/downloads/list}


To check the code out of subversion:\newline
\texttt{svn checkout http://rl-glue-ext.googlecode.com/svn/trunk/projects/codecs/C C-Codec}

\subsection{Installing the Codec}
The codec package was made with autotools, which means that you shouldn't have to do much work to get it installed.  

\subsubsection{Simple Codec Install}
If you are working on your own machine, it is usually easiest to install the headers and libraries into \texttt{/usr/local}, which is the default installation location but requires \textit{sudo} or \textit{root} access.

The steps are:
\begin{verbatim}
	>$ ./configure
	>$ make
	>$ sudo make install
\end{verbatim}

Provided everything goes well, the headers have now been installed to \texttt{/usr/local/include} and the libs to \texttt{/usr/local/lib}.

\subsubsection{Install Codec when RL-Glue is in a custom location}

If \texttt{configure} can't find RL-Glue installed on your machine, it will give you an error like the following:
\begin{verbatim}
checking for rlConnect in -lrlgluenetdev... no
configure: error: RL-Glue library not found.
You must have RL-Glue installed to use this codec. 
	
If you have not downloaded it please see http://glue.rl-community.org/
If you do have it installed in a non-standard location you may need to use the 
--with-rl-glue=/path/to/rlglue 
command line switch to specify where the rl-glue root is located.
\end{verbatim}

If you installed RL-Glue to some place other than \texttt{/usr/local}, say \texttt{/Users/btanner/tmp/rlglue}, you could do:
\begin{verbatim}
	>$ ./configure --with-rl-glue=/Users/btanner/tmp/rlglue
	>$ make
	>$ sudo make install
\end{verbatim}

\subsubsection{Install Codec To Custom Location (without \textit{root} access)}
If you don't have \textit{sudo} or \textit{root} access on the target machine, you can install the codec in your home directory (or other directory you have access to).
If you install to a custom location, you will need set your \texttt{CFLAGS} and \texttt{LDFLAGS} variables appropriately when compiling your agents, environments, and experiments. See Section \ref{sec:custom-flags} for more information.

For example, maybe we want to install the codec to \texttt{/Users/btanner/tmp/rlglue}.  This will \textbf{not} clobber RL-Glue if it is already installed to this location, it will install beside it.  The commands are:
\begin{verbatim}
	>$ ./configure --prefix=/Users/btanner/tmp/rlglue
	>$ make
	>$ make install
\end{verbatim}

Provided everything goes well, the headers and libraries have been respectively installed to\newline
\texttt{/Users/btanner/tmp/rlglue/include} and \texttt{/Users/btanner/tmp/rlglue/lib}.


\subsection{Uninstall}
If you decide that you don't want the RL-Glue C codec on your machine anymore, you can easily uninstall it.  The procedures varies a tiny bit depending on if you installed it to the default location, or somewhere custom.

\subsubsection{Codec Installed To Default Location}
\begin{verbatim}
	>$ ./configure
	>$ sudo make uninstall
\end{verbatim}

This will remove all of the headers and libraries from \texttt{/usr/local}.

\subsubsection{Codec Installed To Custom Location}
You'll need to make sure that either you haven't reconfigured the directory you downloaded from, or, if you removed/changed that already, you have to run configure again the exact same way as when you installed it.  For example:
\begin{verbatim}
	>$ ./configure --prefix=/Users/btanner/tmp/rlglue
	>$ make uninstall
\end{verbatim}

That's it!  This will remove all of the headers and libraries from \texttt{/Users/btanner/tmp/rlglue}.


\section{Agents}
\label{sec:agent}
We have provided a skeleton agent with the codec that is a good starting point for agents that you may write in the future.
It implements all the required functions and provides a good example of how to compile a simple agent.

The pertinent files are:
\begin{verbatim}
	examples/skeleton_agent/SkeletonAgent.c
	examples/skeleton_agent/Makefile
\end{verbatim}

This agent does not learn anything and randomly chooses integer action $0$ or $1$.  

If RL-Glue and this codec have been installed in the default location, \texttt{/usr/local}, then you can compile and run the agent like:
\begin{verbatim}
	>$ cd examples/skeleton_agent
	>$ make
	>$ ./SkeletonAgent
\end{verbatim}

You will see something like:
\begin{verbatim}
	RL-Glue C Agent Codec Version 1.0-alpha-3, Build 192:208M
		Connecting to host=127.0.0.1 on port=4096...
\end{verbatim}

This means that the SkeletonAgent is running, and trying to connect to the \texttt{rl\_glue} executable server on the local machine through port $4096$! 

You can kill the process by pressing \texttt{CTRL-C} on your keyboard.

See Section \ref{sec:custom-flags} if RL-Glue or this Codec are not installed in default locations.

The Skeleton agent is very simple and well documented, so we won't spend any more time talking about it in these instructions.
Please open it up and take a look.

\textbf{POSSIBLE CONTRIBUTION}: If you take a look at the agent and you think it's not easy to understand, think it could be better documented, 
or just that it should do some fancier things, let us know and we'll be happy to do it!

We will spend a little bit talking about how to compile the agent, because not everyone is comfortable with using a \texttt{Makefile}.  To compile
the agent from the command line, you could do:
\begin{verbatim}
	>$ cc SkeletonAgent.c -lrlutils -lrlagent -o SkeletonAgent
\end{verbatim}

On some platforms, you may need to add \texttt{-lrlgluenetdev}

It might be useful to break this down a little bit:
\begin{description}
\item [cc] The C compiler.  You could also use \texttt{gcc} or \texttt{g++}, etc.
\item [SkeletonAgent.c] Compile the SkeletonAgent.c source file.
\item [-lrlutils] Link to the RLUtils library, which comes with the RL-Glue project.  That library contains convenience functions for allocating and cleaning up the structure types.  If you 
don't use these convenience functions, you don't need this library.  See the RL-Glue 3.0 technical manual for more information about RLUtils.
\item [-lrlagent] Link to the RLAgent library of the codec.  This is where the main agent loop is defined. The main agent loop connects to the \texttt{rl\_glue} executable server and dispatches commands sent by the glue.
\item [-lrlgluenetdev] Link to the RLGlueNetDev library from the RL-Glue project.  This library is automatically linked through \texttt{rlagent} on most platform (except notably Cygwin).  
RLGlueNetDev provides implementations of the low level network code that is used by all three parts of the codec, as well as the \texttt{rl\_glue} executable server.
\end{description}

%Find a place for this
\subsection{Custom Flags for Custom Installs}
\label{sec:custom-flags}
If RL-Glue \textbf{or} this codec have been installed in a custom location (for example: \texttt{/Users/joe/glue}), then you will
need to set the header search path in \texttt{CFLAGS} and the library search path in \texttt{LDFLAGS}.  You can either do this each time you call make, 
or you can export the values as environment variables.  These instructions apply to agents, environments, and experiment programs.

To do it on the command line:
\begin{verbatim}
>$ CFLAGS=-I/Users/joe/glue/include LDFLAGS=-L/Users/joe/glue/lib make
\end{verbatim}

That might turn out to be quite a hassle while you are developing.  In that case, you can either update the \texttt{Makefile} to include these flags, 
or set an environment variable.  If you are using the bash shell you can \texttt{export} the environment variables:
\begin{verbatim}
>$ export CFLAGS=-I/Users/joe/glue/include
>$ export LDFLAGS=-L/Users/joe/glue/lib
>$ make
\end{verbatim}

In some cases, you may be able to compile and link your programs without incident, but you receive shared library loading errors when you try to execute them, as mentioned in Gotchas! (Section \ref{sec:gotchas-shared-libs}).

The reason for these errors is that the search path of the loader does not include \texttt{/usr/local/lib}.  This problem has both temporary and permanent fixes.

To temporarily fix the problem, you can set \texttt{LD\_LIBRARY\_PATH} (Linux) or \texttt{DYLD\_LIBRARY\_PATH} (OS X) environment variables, like:
\begin{verbatim}
	>$ export LD_LIBRARY_PATH=/Users/joe/glue/lib
\end{verbatim}

In some cases (64-bit linux) you may have to use this approach even when RL-Glue and this codec are installed in the default locations:
\begin{verbatim}
	>$ export LD_LIBRARY_PATH=/usr/local/lib
\end{verbatim}

When you open a new terminal window, all of these environment variables will be lost unless you put the appropriate \texttt{export} lines in your shell startup script.

The permanent solution requires \texttt{root} or \texttt{sudo} access to the machine.  You can  edit the file \texttt{/etc/ld.so.conf} and add a line like: \texttt{/usr/local/lib} to the file.  Then, if you call \texttt{sudo ldconfig}, the loader will know to look there for libraries in the future.

\section{Environments}
We have provided a skeleton environment with the codec that is a good starting point for environments that you may write in the future.
It implements all the required functions and provides a good example of how to compile a simple environment.  This section will follow the same 
pattern as the agent version (Section \ref{sec:agent}).  This section will be less detailed because many ideas are similar or identical.

The pertinent files are:
\begin{verbatim}
	examples/skeleton_environment/SkeletonEnvironment.c
	examples/skeleton_environment/Makefile
\end{verbatim}

This environment is episodic, with 21 states, labeled $\{0, 1,\ldots,19,20\}$. States $\{0, 20\}$ are terminal and return rewards of $\{-1, +1\}$ respectively.  The other states return reward of $0$.
There are two actions, $\{0, 1\}$.  Action $0$ decrements the state number, and action $1$ increments it. The environment starts in state 10.

If RL-Glue and this codec have been installed in the default location, \texttt{/usr/local}, then you can compile and run the environment like:
\begin{verbatim}
	>$ cd examples/skeleton_environment
	>$ make
	>$ ./SkeletonEnvironment
\end{verbatim}

You will see something like:
\begin{verbatim}
	RL-Glue C Environment Codec Version 1.0-alpha-3, Build 192:208M
		Connecting to host=127.0.0.1 on port=4096...
\end{verbatim}

This means that the SkeletonEnvironment is running, and trying to connect to the \texttt{rl\_glue} executable server on the local machine through port $4096$! 

You can kill the process by pressing \texttt{CTRL-C} on your keyboard.

See Section \ref{sec:custom-flags} if RL-Glue or this Codec are not installed in default locations.

The Skeleton environment is very simple and well documented, so we won't spend any more time talking about it in these instructions.
Please open it up and take a look.

\textbf{POSSIBLE CONTRIBUTION}: If you take a look at the environment and you think it's not easy to understand, think it could be better documented, 
or just that it should do some fancier things, let us know and we'll be happy to do it!

Compiling the environment is almost identical to compiling the skeleton agent, except you need to link to the \texttt{RLEnvironment} library instead of \texttt{RLAgent}.
\begin{verbatim}
	>$ cc SkeletonEnvironment.c -lrlutils -lrlenvironment -o SkeletonEnvironment
\end{verbatim}

On some platforms, you may need to add \texttt{-lrlgluenetdev}

\section{Experiments}
We have provided a skeleton experiment with the codec that is a good starting point for experiment that you may write in the future.
It implements all the required functions and provides a good example of how to compile a simple experiment.  This section will follow the same 
pattern as the agent version (Section \ref{sec:agent}).  This section will be less detailed because many ideas are similar or identical.

The pertinent files are:
\begin{verbatim}
	examples/skeleton_experiment/SkeletonExperiment.c
	examples/skeleton_experiment/Makefile
\end{verbatim}

This experiment runs \texttt{RL\_Episode} a few times, sends some messages to the agent and environment, and then steps through one episode using \texttt{RL\_step}.

If RL-Glue and this codec have been installed in the default location, \texttt{/usr/local}, then you can compile and run the experiment like:
\begin{verbatim}
	>$ cd examples/skeleton_experiment
	>$ make
	>$ ./SkeletonExperiment
\end{verbatim}

You will see something like:
\begin{verbatim}
	RL-Glue C Experiment Codec Version 1.0-alpha-3, Build 192:208M
		Connecting to host=127.0.0.1 on port=4096...
\end{verbatim}

This means that the SkeletonExperiment is running, and trying to connect to the \texttt{rl\_glue} executable server on the local machine through port $4096$!  

You can kill the process by pressing \texttt{CTRL-C} on your keyboard.

See Section \ref{sec:custom-flags} if RL-Glue or this Codec are not installed in default locations.

The Skeleton experiment is very simple and well documented, so we won't spend any more time talking about it in these instructions.
Please open it up and take a look.

\textbf{POSSIBLE CONTRIBUTION}: If you take a look at the experiment and you think it's not easy to understand, think it could be better documented, 
or just that it should do some fancier things, let us know and we'll be happy to do it!

Compiling the experiment is almost identical to compiling the skeleton agent, except you need to link to the \texttt{RLExperiment} library instead of \texttt{RLAgent}.
\begin{verbatim}
	>$ cc SkeletonExperiment.c -lrlutils -lrlexperiment -o SkeletonExperiment
\end{verbatim}

On some platforms, you may need to add \texttt{-lrlgluenetdev}

\subsection{Gotchas!}
\subsubsection{Crashes and Bus Errors in Experiment Program}
If you are running an experiment using \texttt{RL\_step}, beware that the last step (when \texttt{terminal==1}), the action will be empty.  If you try to access the values of the actions in this 
case, you may crash your program.

\subsubsection{Shared Library Loading Errors}
\label{sec:gotchas-shared-libs}
On some machines we've used, the codec installs without incident, but when the agent/environment/experiment is run, the system gives an error message similar to:
\begin{verbatim}
	>$ ./SkeletonAgent: error while loading shared libraries: librlagent-1:0:0.so.1:
	cannot open shared object file: No such file or directory
\end{verbatim}

So far this has only happened on 64-bit Linux, both a machine at the University
and a personal install of 64-bit Ubuntu, Hardy Heron.

The reason for the error is that \texttt{/usr/local/lib} is not in the the loader's search path.  If this happens, the operating system might have an alternate search path, and might not be looking in \texttt{/usr/local/lib} for libraries. 
You can troubleshoot this problem by doing:
\begin{verbatim}
	>$ LD_DEBUG=libs ./RL_agent
\end{verbatim}
If you see that \texttt{/usr/local/lib} is not in the search path, you may want to add it to your library search path using \texttt{LDFLAGS} or \texttt{LD\_LIBRARY\_PATH}.  See Section \ref{sec:custom-flags} for more information.  

\section{Putting it all together}
At this point, we've compiled and run each of the three components, now it's time to run them with the \texttt{rl\_glue} executable server.  The following will work from the examples directory if you have them all built, and RL-Glue installed in the default location:
\begin{verbatim}
	>$ rl_glue &
	>$ skeleton_agent/SkeletonAgent &
	>$ skeleton_environment/SkeletonEnvironment &
	>$ skeleton_experiment/SkeletonExperiment	
\end{verbatim}

If RL-Glue is not installed in the default location, you'll have to start the \texttt{rl\_glue} executable server using it's full path (unless it's in your \texttt{PATH} environment variable):
\begin{verbatim}
	>$ /path/to/rl-glue/bin/rl_glue &
\end{verbatim}

You should see output like the following if it worked:
\begin{verbatim}
	>$ rl_glue &
	RL-Glue Version 3.0-alpha-3, Build 848:852M
	RL-Glue is listening for connections on port=4096

	>$ skeleton_agent/SkeletonAgent & 
	RL-Glue C Agent Codec Version 1.0-alpha-3, Build 192:208M
		Connecting to host=127.0.0.1 on port=4096...
		RL-Glue C Agent Codec :: Connected
		RL-Glue :: Agent connected.

	>$ skeleton_environment/SkeletonEnvironment & 
	RL-Glue C Environment Codec Version 1.0-alpha-3, Build 192:208M
		Connecting to host=127.0.0.1 on port=4096...
		RL-Glue C Environment Codec :: Connected
		RL-Glue :: Environment connected.

	$> skeleton_experiment/SkeletonExperiment


	Experiment starting up!
	RL-Glue C Experiment Codec Version 1.0-alpha-3, Build 192:208M
		Connecting to host=127.0.0.1 on port=4096...
		RL-Glue C Experiment Codec :: Connected
		RL-Glue :: Experiment connected.
	RL_init called, the environment sent task spec: 2:e:1_[i]_[0,5]:1_[i]_[0,1]:[-1,1]


	----------Sending some sample messages----------
	Agent responded to "what is your name?" with: 
	my name is skeleton_agent!
	Agent responded to "If at first you don't succeed; call it version 1.0" with: 
	I don't know how to respond to your message

	Environment responded to "what is your name?" with: 
	my name is skeleton_environment!
	Environment responded to "If at first you don't succeed; call it version 1.0" with: 
	I don't know how to respond to your message


	----------Running a few episodes----------
	Episode 0	 100 steps 	0.000000 total reward	 0 natural end 
	Episode 1	 44 steps 	-1.000000 total reward	 1 natural end 
	Episode 2	 18 steps 	-1.000000 total reward	 1 natural end 
	Episode 3	 100 steps 	0.000000 total reward	 0 natural end 
	Episode 4	 50 steps 	1.000000 total reward	 1 natural end 
	Episode 5	 1 steps 	0.000000 total reward	 0 natural end 
	Episode 6	 28 steps 	1.000000 total reward	 1 natural end 


	----------Stepping through an episode----------
	First observation and action were: 10 1

	----------Summary----------
	It ran for 144 steps, total reward was: -1.000000
	
\end{verbatim}

\section{Codec Specification, Memory Allocation, Types, and Function Prototypes}
There is some important information about memory management practices in RL-Glue, as well as detailed information about the C/C++ RL-Glue type definitions available in the 
RL-Glue 3.0 technical manual.  We've opted to not duplicate those 6 pages of useful information here : but you really should go read them.

The main thing to realize is that agents, environments, and experiment programs are completely interchangeable between \textit{direct-compile} (RL-Glue project) and \textit{socket-communication}
(this codec).  An agent written for either one will be compiled the same way, the only difference is what library it is linked to.  So, 95\% of the technical manual from the RL-Glue project applies 
perfectly to people using this codec! 




\section{Advanced Features}
This section will explain how to set custom target IP addresses (to connect over the network) and custom ports (to run multiple experiments on one machine or to avoid firewall issues).

Someone should write this later (Opportunity to contribute!).

\section{Changes and 2.x Backward Compatibility}
There were many changes from RL-Glue 2.x to RL-Glue 3.x.  Most of them are at the level of the API and project organization, and are addressed in the \htmladdnormallink{RL-Glue overview documentation}{http://rl-glue.googlecode.com/svn/trunk/docs/html/index.html}.

\subsection{Types}
All of the types that existed in the 2.x codec (ex: \texttt{Observation} instead of \texttt{observation\_t}) are still supported through a definition file called 
\texttt{legacy\_types.h}.  If you don't want to update your old agents to the new types, you can use the old names by doing the following in your source files:
\begin{verbatim}
	#include<rlglue/legacy_types.h>
\end{verbatim}



\section{Frequently Asked Questions}
We're waiting to hear your questions!

\section{Credits and Acknowledgements}
Andrew Butcher originally wrote the RL-Glue network library and first version of this codec.  Thanks Andrew.

Brian Tanner has since grabbed the torch and has continued to develop the codec.

\subsection{Contributing}
If you would like to become a member of this project and contribute updates/changes to the code, please send a message to rl-glue@googlegroups.com.


\section*{Document Information}
\begin{verbatim}
Revision Number: $Rev$
Last Updated By: $Author$
Last Updated   : $Date$
$URL$
\end{verbatim}

\end{document} 
